\documentclass[11pt,]{article}
\usepackage{lmodern}
\usepackage{amssymb,amsmath}
\usepackage{ifxetex,ifluatex}
\usepackage{fixltx2e} % provides \textsubscript
\ifnum 0\ifxetex 1\fi\ifluatex 1\fi=0 % if pdftex
  \usepackage[T1]{fontenc}
  \usepackage[utf8]{inputenc}
\else % if luatex or xelatex
  \ifxetex
    \usepackage{mathspec}
  \else
    \usepackage{fontspec}
  \fi
  \defaultfontfeatures{Ligatures=TeX,Scale=MatchLowercase}
\fi
% use upquote if available, for straight quotes in verbatim environments
\IfFileExists{upquote.sty}{\usepackage{upquote}}{}
% use microtype if available
\IfFileExists{microtype.sty}{%
\usepackage{microtype}
\UseMicrotypeSet[protrusion]{basicmath} % disable protrusion for tt fonts
}{}
\usepackage[margin=1in]{geometry}
\usepackage{hyperref}
\hypersetup{unicode=true,
            pdftitle={The influence of female secondary school enrollment on total fertility rate at the global and country level},
            pdfauthor={Tauriq Jamalie (3437177), Adnaan Emandien (3540482), Hafeez Sonday (3440252)},
            pdfborder={0 0 0},
            breaklinks=true}
\urlstyle{same}  % don't use monospace font for urls
\usepackage{graphicx,grffile}
\makeatletter
\def\maxwidth{\ifdim\Gin@nat@width>\linewidth\linewidth\else\Gin@nat@width\fi}
\def\maxheight{\ifdim\Gin@nat@height>\textheight\textheight\else\Gin@nat@height\fi}
\makeatother
% Scale images if necessary, so that they will not overflow the page
% margins by default, and it is still possible to overwrite the defaults
% using explicit options in \includegraphics[width, height, ...]{}
\setkeys{Gin}{width=\maxwidth,height=\maxheight,keepaspectratio}
\IfFileExists{parskip.sty}{%
\usepackage{parskip}
}{% else
\setlength{\parindent}{0pt}
\setlength{\parskip}{6pt plus 2pt minus 1pt}
}
\setlength{\emergencystretch}{3em}  % prevent overfull lines
\providecommand{\tightlist}{%
  \setlength{\itemsep}{0pt}\setlength{\parskip}{0pt}}
\setcounter{secnumdepth}{0}
% Redefines (sub)paragraphs to behave more like sections
\ifx\paragraph\undefined\else
\let\oldparagraph\paragraph
\renewcommand{\paragraph}[1]{\oldparagraph{#1}\mbox{}}
\fi
\ifx\subparagraph\undefined\else
\let\oldsubparagraph\subparagraph
\renewcommand{\subparagraph}[1]{\oldsubparagraph{#1}\mbox{}}
\fi

%%% Use protect on footnotes to avoid problems with footnotes in titles
\let\rmarkdownfootnote\footnote%
\def\footnote{\protect\rmarkdownfootnote}

%%% Change title format to be more compact
\usepackage{titling}

% Create subtitle command for use in maketitle
\newcommand{\subtitle}[1]{
  \posttitle{
    \begin{center}\large#1\end{center}
    }
}

\setlength{\droptitle}{-2em}
  \title{The influence of female secondary school enrollment on total fertility
rate at the global and country level}
  \pretitle{\vspace{\droptitle}\centering\huge}
  \posttitle{\par}
  \author{Tauriq Jamalie (3437177), Adnaan Emandien (3540482), Hafeez Sonday
(3440252)}
  \preauthor{\centering\large\emph}
  \postauthor{\par}
  \predate{\centering\large\emph}
  \postdate{\par}
  \date{29 June 2018}


\begin{document}
\maketitle

\section{Abstract}\label{abstract}

\section{Introduction}\label{introduction}

Does female secondary school education lower fertility? This is an
important question as both education and total fertility rates are
considered important processes for economic development especially in
countries transitioning from developing to developed (Chisadza \&
Bittencourt, 2015). The effect of education on fertility is complex and
has been studied extensively. The topic is highly relevant in research
on reproductive behaviour (Testa, 2014).

Modern developed economies are typically characterized by increased
human capital accumulation, decreased fertility rates as well as high
levels of productivity (Chisadza \& Bittencourt, 2015). Previous studies
on secondary education levels have shown a consistent negative
relationship with fertility (Chisadza \& Bittencourt, 2015). This has
been strongly observed across regions and time (Kim, 2016). Previous
studies indicate that higher levels of education are significant in
reducing fertility within a region (Chisadza \& Bittencourt, 2015).
These studies provide evidence for economies that are entering their own
demographic transitions in moving from stagnation to economic growth
(Chisadza \& Bittencourt, 2015). A woman's education level may affect
fertility via its impact on a woman's health in conjunction with its
impact on a woman's physical capacity to give birth. Furthermore, a
woman's education level could affect fertility through its impact on
children's health, the number of children desired as well as a woman's
ability to effectively plan families and knowledge on birth control
methods (Kim, 2016). These mechanisms are dependent on the individual's
circumstances as well as the circumstances experienced by the
institution and country.

Within modern societies several explanations have been reviewed in
triggering the decline in fertility rates (Chisadza \& Bittencourt,
2015). Consider the Barro-Becker theory that focuses on the opportunity
costs that's involved with the growing income per capita which may cause
parents to substitute the quantity of children for higher quality
(Chisadza \& Bittencourt, 2015). The unified growth theory emphasizes
technologies role in encouraging the investments in children's education
(Chisadza \& Bittencourt, 2015). The decrease in the gender gap which
ultimately raises the cost of children may also account for the results
attained (Chisadza \& Bittencourt, 2015). There may also be a shift in
traditions with regards to the old-age security hypothesis that deems
younger generations a security measure for older generations (Chisadza
\& Bittencourt, 2015). Finally, mortality rates are declining- this
reduces the need to conceive more children as there is less requirement
to replace offspring that do not survive (Chisadza \& Bittencourt,
2015).

Differences in fertility decline timings gave rise to the difference in
take-offs of the demographic transitions which subsequently led to the
variation in the levels of economic development which may be observed
between developing economies and developed economies.\\
The null hypothesis states that female secondary school enrolment will
have no effect on total fertility rate of a population, while the
alternative hypothesis states that female secondary education enrolment
will influence total fertility rate of a population.\\
The aim of this assignment is to examine the influence female secondary
school enrolment has on total fertility rate at both the global and
country level using the latest available data.

\section{Methods and Materials}\label{methods-and-materials}

\subsection{Setting-up and loading the
data}\label{setting-up-and-loading-the-data}

Datasets relating to worldwide fertility and percentage of females
enrolled in secondary school, as well as Kenyan fertility and percentage
of females enrolled in secondary schools in Kenya was downloaded from
The World Bank (ref) and UNESCO (org).Datasets were then cleaned as it
contained unnecessary detail that did not align to our study's aims as
well as removing entries with missing information. The data was
subsequently converted into a comma separated value (csv) file in
Microsoft excel in order to be read in Rstudio. The datasets contained
the percentage of females enrolled in secondary school and average
number of children per woman of 151 different countries across the
world. Similarly the kenya data containing percentage of females
enrolled in secondary school and average number of children per woman
was cleaned, converted to a csv and subsequently loaded in `Rstudio'
where the packages `tidyverse', `ggpubr', and `corrplot' were set up in
order to read the data, and the data was loaded. These steps can be seen
below:

\subsection{Assumptions}\label{assumptions}

The assumptions for this data were checked. Specifically whether the the
dependent variable was continuous, the observations in the groups being
compared were independent of each other, the data are normally
distributed, and that the data are homoscedastic, and that there are no
outliers. Normality and homoscedasticity was checked using the
Shapiro-Wilk test and variance. This was done using the code below:

\subsubsection{Normality and Homoscedasticity of Global
Data}\label{normality-and-homoscedasticity-of-global-data}

\subsubsection{Normality and Homoscedasticity of Kenya
Data}\label{normality-and-homoscedasticity-of-kenya-data}

\subsection{Simple linear regression and
correlation}\label{simple-linear-regression-and-correlation}

A linear model was then fitted for the global data and Kenya data to
carry out regression analysis. A graph showing the correlation between
percentage of female secondary school enrolment and total fertility
rates of every country was produced using the `ggplot' function in the
package `tidyverse'.

The function `coefficients' was used in order to extract model
coefficients from linear models of global data and kenya.

Finally, a correlation test using Kendall's rank correlation tau was
performed for the percentage of females enrolled in secondary school and
fertility rate of the global data using the `cor.test' function.
Similarly a Pearsons test was done on the Kenya data. This code can be
seen below.

\subsubsection{Simple linear regression of global
data}\label{simple-linear-regression-of-global-data}

\subsubsection{A graph of the linear regression of the global
data}\label{a-graph-of-the-linear-regression-of-the-global-data}

\subsubsection{Kendall Correlation of the global
data}\label{kendall-correlation-of-the-global-data}

\subsubsection{Simple linear regression of the Kenya
data}\label{simple-linear-regression-of-the-kenya-data}

\subsubsection{A graph of the linear regression of the Kenya
data}\label{a-graph-of-the-linear-regression-of-the-kenya-data}

\subsubsection{Pearsons Correlation of the Kenya
data}\label{pearsons-correlation-of-the-kenya-data}

\section{Result}\label{result}

Both tests revealed a p-value less than 0.05 (alpha \textless{}0.05),
indicating that at a global level the data is not normally distributed.
However, at a country level i.e.~Kenya, the Shapiro-Wilk tests revealed
a p-value of 0.23 for percentage of females enrolled in secondary
education and 0.88 for Kenya's fertility rate, indicating that the
country level dataset is in fact normally distributed. this can be seen
below.

\subsubsection{Normality of the avergage number of children per woman
within the Global
Data}\label{normality-of-the-avergage-number-of-children-per-woman-within-the-global-data}

\begin{verbatim}
## 
##  Shapiro-Wilk normality test
## 
## data:  Global_correlation$Percent_enrolled
## W = 0.86695, p-value = 2.394e-10
\end{verbatim}

\subsubsection{Normality of the avergage number of children per woman
within the Global
Data}\label{normality-of-the-avergage-number-of-children-per-woman-within-the-global-data-1}

\begin{verbatim}
## 
##  Shapiro-Wilk normality test
## 
## data:  Global_correlation$TFR
## W = 0.86872, p-value = 2.903e-10
\end{verbatim}

\subsubsection{The Homoscedasticity of the percent of females endrolled
in secondary education in the Global
Data}\label{the-homoscedasticity-of-the-percent-of-females-endrolled-in-secondary-education-in-the-global-data}

\begin{verbatim}
## [1] 733.2138
\end{verbatim}

\subsubsection{The Homoscedasticity of the avergage number of children
per woman the Global
Data}\label{the-homoscedasticity-of-the-avergage-number-of-children-per-woman-the-global-data}

\begin{verbatim}
## [1] 1.945883
\end{verbatim}

\subsubsection{Normality of the avergage number of children per woman
within the Kenya
Data}\label{normality-of-the-avergage-number-of-children-per-woman-within-the-kenya-data}

\begin{verbatim}
## 
##  Shapiro-Wilk normality test
## 
## data:  kenya_combo$Percent_enrollment
## W = 0.90858, p-value = 0.2346
\end{verbatim}

\subsubsection{Normality of the avergage number of children per woman
within the Kenya
Data}\label{normality-of-the-avergage-number-of-children-per-woman-within-the-kenya-data-1}

\begin{verbatim}
## 
##  Shapiro-Wilk normality test
## 
## data:  kenya_combo$TFR
## W = 0.96981, p-value = 0.8846
\end{verbatim}

\subsubsection{The Homoscedasticity of the percent of females endrolled
in secondary education in the Kenya
Data}\label{the-homoscedasticity-of-the-percent-of-females-endrolled-in-secondary-education-in-the-kenya-data}

\begin{verbatim}
## [1] 47.89761
\end{verbatim}

\subsubsection{The Homoscedasticity of the avergage number of children
per woman in the Kenya
Data}\label{the-homoscedasticity-of-the-avergage-number-of-children-per-woman-in-the-kenya-data}

\begin{verbatim}
## [1] 0.0612222
\end{verbatim}

The test divulged a tau value of 0.53 indicating the presence of a
relationship between the two aforementioned variables. Whether the
relationship is negative or positive is irrelevant. The data were then
subjected to a regression analysis. What is important to note here is
that the results display an r² value of 0.72, indicative of a very
strong relationship between the two variables again.

\subsubsection{Simple linear regression of global
data}\label{simple-linear-regression-of-global-data-1}

\begin{verbatim}
## 
## Call:
## lm(formula = Percent_enrolled ~ TFR, data = Global_correlation)
## 
## Residuals:
##     Min      1Q  Median      3Q     Max 
## -34.370  -8.623   0.558   8.521  41.307 
## 
## Coefficients:
##             Estimate Std. Error t value Pr(>|t|)    
## (Intercept) 113.7353     2.5891   43.93   <2e-16 ***
## TFR         -16.5220     0.8348  -19.79   <2e-16 ***
## ---
## Signif. codes:  0 '***' 0.001 '**' 0.01 '*' 0.05 '.' 0.1 ' ' 1
## 
## Residual standard error: 14.26 on 149 degrees of freedom
## Multiple R-squared:  0.7245, Adjusted R-squared:  0.7226 
## F-statistic: 391.7 on 1 and 149 DF,  p-value: < 2.2e-16
\end{verbatim}

\subsubsection{Kendall Correlation of the global
data}\label{kendall-correlation-of-the-global-data-1}

\begin{verbatim}
## 
##  Kendall's rank correlation tau
## 
## data:  Global_correlation$Percent_enrolled and Global_correlation$TFR
## z = -9.7068, p-value < 2.2e-16
## alternative hypothesis: true tau is not equal to 0
## sample estimates:
##        tau 
## -0.5330448
\end{verbatim}

\subsubsection{Simple linear regression of the Kenya
data}\label{simple-linear-regression-of-the-kenya-data-1}

\begin{verbatim}
## 
## Call:
## lm(formula = Percent_enrollment ~ TFR, data = kenya_combo)
## 
## Residuals:
##     Min      1Q  Median      3Q     Max 
## -2.4320 -0.7194  0.3021  0.4399  2.6236 
## 
## Coefficients:
##             Estimate Std. Error t value Pr(>|t|)    
## (Intercept)  179.787      9.063   19.84 9.75e-09 ***
## TFR          -27.414      1.851  -14.81 1.26e-07 ***
## ---
## Signif. codes:  0 '***' 0.001 '**' 0.01 '*' 0.05 '.' 0.1 ' ' 1
## 
## Residual standard error: 1.448 on 9 degrees of freedom
## Multiple R-squared:  0.9606, Adjusted R-squared:  0.9562 
## F-statistic: 219.4 on 1 and 9 DF,  p-value: 1.259e-07
\end{verbatim}

\subsubsection{Pearsons Correlation of the Kenya
data}\label{pearsons-correlation-of-the-kenya-data-1}

\begin{verbatim}
## 
##  Pearson's product-moment correlation
## 
## data:  kenya_combo$Percent_enrollment and kenya_combo$TFR
## t = -14.812, df = 9, p-value = 1.259e-07
## alternative hypothesis: true correlation is not equal to 0
## 95 percent confidence interval:
##  -0.9949855 -0.9227339
## sample estimates:
##        cor 
## -0.9800994
\end{verbatim}

\section{Discussion}\label{discussion}

In this study we have shown that education does indeed have an effect on
total fertility rates, in that as the percentage of females enrolled in
secondary education (FESE) increases so the total fertility rate (TFR)
decreases. This is true at both a global level as well as a country
specific level as was demonstrated. However, as can be seen in figure
one and two the strength of the relationship between the two variables,
FESE and TFR differ. This is to be expected as the global data includes
a wider range whereas at a country level populations are much smaller.
Nevertheless the degree of correlation is still highly profound.

In 2014, Manoel Bittencourt of the University of Pretoria conducted a
similar study. He however made use of primary school completion instead.
Nonetheless our results are on par with that of his study, in that they
display a relationship in which education can lower fertility rates.
Furthermore these findings are important in that lower fertility rates
as a result of a more educated population, leads to an increase in
capital per wage earner resulting in increased productivity and
subsequently increased growth rates (Bittencourt, 2014). In the case of
underdeveloped countries, this means that the country has the ability to
shift from a state in which the population increases at a faster rate
than its means of subsistence to a more sustained growth system.

A sustainable growth system does sound appealing but how can one enforce
the attainment of education? Consider the case of the Kenyan educational
reform of 1985 where the government added an extra year before being
suitable to earn the certificate of primary education. With Kenya's
independence from colonialism, came the need for an educational reform
mainly due to the fact that prior to its independence the Kenyan
education system was extremely basic and lacked substantial content that
promotes widespread sustainable employment (Wanjohi, 2011). In 2012,
Chicoine conducted a study targeting data collected during the time of
this reform as well as post reform data in which he tested its effect on
fertility rates. Much like Bittencourt as well as the results related to
the present study, he concluded that as a result of an additional one
year to primary education, a significant decrease in total fertility
rate was observed as a result of postponed marriage and sexual activity,
hence increased schooling can have a positive effect on choices and
decision making of young women. Indicating the validity of this method
in terms of a solution in mitigate an increasing population size.

\section{References}\label{references}

Basu AM. 2002. Why does Education Lead to Lower Fertility? A Critical
Review of Some of the Possibilities. World Development 30: 1779-1790.

Brown TC, Bergstrom JC, Loomis JB. 2007. Defining, Valuing, and
Providing Ecosystem Goods and Services. Ecosystem Goods \& Services:
1-48.

Brown TC, Bergstrom JC, Loomis JB. 2007. Defining, Valuing, and
Providing Ecosystem Goods and Services. Ecosystem Goods \& Services:
1-48.

Chicoine L. 2012. Education and Fertility: Evidence from a Policy Change
in Kenya. IZA Discussion Paper 6778: 1-63.

Chisadza C, Bittencourt M. 2015. Education and Fertility: Panel Evidence
fom sub-Saharan African. University of Pretoria. Department of Economics
Working Paper Series: 1-31.

Cincotta RP, Engelman R. 2000. Nature's place: human population and the
future of biological diversity. Washington DC: Population of Action
International. 87 p.

Edet SIE, Samuel NE, Etim AE, Titus EE. 2014. Impact of Overpopulation
on the Biological Diversity Conservation in Boki Local Government Area
of Cross River State, Nigeria. American Journal of Environmental
Engineering 4: 94-98.

Isaacs L. Appeal to preserve aquifer. 2018. IOL. Available at
\url{https://www.iol.co.za/capetimes/news/appeal-to-preserve-aquifer-13617553}
{[}accessed at 18 June 2018{]}.

Kim J. 2016. Female education and its impact on fertility. The
relationship is more complex than one might think. IZA World of Labor
2016 228: 1-10.

Overpopulation and Biodiversity. 2014. Biodiversity. Available at
\url{http://overpopulationbiodiversity.weebly.com/biodiversity.html}
{[}accessed 18 June 2018{]}. Ryerson WN. 2010. Population: The
Multiplier of Everything Else. The Post Carbon Reader Series:
Population: 1-20.

Sustainable Development Goals. Educational Attainment, Population by
completed level of education. Available at
\url{http://data.uis.unesco.org/} {[}accessed 16 June 2018{]}.

The World Bank Group. 2018. Fertility rate, total (births per woman).
Available at \url{https://data.worldbank.org/indicator/SP.DYN.TFRT.IN}
{[}accessed 14 June 2018{]}.

The World Bank Group. 2018. School enrolment, secondary (\% gross).
Available at
\url{https://data.worldbank.org/indicator/SE.SEC.ENRR?view=chart}
{[}accessed 14 June 2018{]}.

Wanjohi AM. 2011. Development of Education System in Kenya since
Independence. KENPRO Online Papers Portal. Available at
\url{http://www.kenpro.org/papers/education-system-kenya-independence.htm}
{[}accessed 16 June 2018{]}.

Why population matters to biodiversity. 2012. PAI. Available at
\url{https://pai.org/policy-briefs/why-population-matters-to-biodiversity/}
{[}accessed at 16 June 2018{]}.

\section{Appendix}\label{appendix}

\includegraphics{R_Markdown_files/figure-latex/unnamed-chunk-22-1.pdf}

\includegraphics{R_Markdown_files/figure-latex/unnamed-chunk-23-1.pdf}


\end{document}
